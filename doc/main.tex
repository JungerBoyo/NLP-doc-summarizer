\documentclass{article}

\usepackage[letterpaper,top=2cm,bottom=2cm,left=1cm,right=1cm,marginparwidth=1.75cm]{geometry}
\usepackage{amsmath}
\usepackage{graphicx}
\usepackage[colorlinks=true, allcolors=blue]{hyperref}
\usepackage{lmodern}  % for bold teletype font
\usepackage{amsmath}  % for \hookrightarrow
\usepackage{xcolor}   % for \textcolor
\usepackage{listings}
\lstset{
  basicstyle=\ttfamily,
  columns=fullflexible,
  frame=single,
  breaklines=true,
  postbreak=\mbox{\textcolor{red}{$\hookrightarrow$}\space},
  inputencoding = utf8,  % Input encoding
  extendedchars = true,  % Extended ASCII
  literate      =        % Support additional characters
      {á}{{\'a}}1  {é}{{\'e}}1  {í}{{\'i}}1 {ó}{{\'o}}1  {ú}{{\'u}}1
      {Á}{{\'A}}1  {É}{{\'E}}1  {Í}{{\'I}}1 {Ó}{{\'O}}1  {Ú}{{\'U}}1
      {à}{{\`a}}1  {è}{{\`e}}1  {ì}{{\`i}}1 {ò}{{\`o}}1  {ù}{{\`u}}1
      {À}{{\`A}}1  {È}{{\`E}}1  {Ì}{{\`I}}1 {Ò}{{\`O}}1  {Ù}{{\`U}}1
      {ä}{{\"a}}1  {ë}{{\"e}}1  {ï}{{\"i}}1 {ö}{{\"o}}1  {ü}{{\"u}}1
      {Ä}{{\"A}}1  {Ë}{{\"E}}1  {Ï}{{\"I}}1 {Ö}{{\"O}}1  {Ü}{{\"U}}1
      {â}{{\^a}}1  {ê}{{\^e}}1  {î}{{\^i}}1 {ô}{{\^o}}1  {û}{{\^u}}1
      {Â}{{\^A}}1  {Ê}{{\^E}}1  {Î}{{\^I}}1 {Ô}{{\^O}}1  {Û}{{\^U}}1
      {œ}{{\oe}}1  {Œ}{{\OE}}1  {æ}{{\ae}}1 {Æ}{{\AE}}1  {ß}{{\ss}}1
      {ẞ}{{\SS}}1  {ç}{{\c{c}}}1 {Ç}{{\c{C}}}1 {ø}{{\o}}1  {Ø}{{\O}}1
      {å}{{\aa}}1  {Å}{{\AA}}1  {ã}{{\~a}}1  {õ}{{\~o}}1 {Ã}{{\~A}}1
      {Õ}{{\~O}}1  {ñ}{{\~n}}1  {Ñ}{{\~N}}1  {¿}{{?`}}1  {¡}{{!`}}1
      {„}{\quotedblbase}1 {“}{\textquotedblleft}1 {–}{$-$}1
      {°}{{\textdegree}}1 {º}{{\textordmasculine}}1 {ª}{{\textordfeminine}}1
      {£}{{\pounds}}1  {©}{{\copyright}}1  {®}{{\textregistered}}1
      {«}{{\guillemotleft}}1  {»}{{\guillemotright}}1  {Ð}{{\DH}}1  {ð}{{\dh}}1
      {Ý}{{\'Y}}1    {ý}{{\'y}}1    {Þ}{{\TH}}1    {þ}{{\th}}1    {Ă}{{\u{A}}}1
      {ă}{{\u{a}}}1  {Ą}{{\k{A}}}1  {ą}{{\k{a}}}1  {Ć}{{\'C}}1    {ć}{{\'c}}1
      {Č}{{\v{C}}}1  {č}{{\v{c}}}1  {Ď}{{\v{D}}}1  {ď}{{\v{d}}}1  {Đ}{{\DJ}}1
      {đ}{{\dj}}1    {Ė}{{\.{E}}}1  {ė}{{\.{e}}}1  {Ę}{{\k{E}}}1  {ę}{{\k{e}}}1
      {Ě}{{\v{E}}}1  {ě}{{\v{e}}}1  {Ğ}{{\u{G}}}1  {ğ}{{\u{g}}}1  {Ĩ}{{\~I}}1
      {ĩ}{{\~\i}}1   {Į}{{\k{I}}}1  {į}{{\k{i}}}1  {İ}{{\.{I}}}1  {ı}{{\i}}1
      {Ĺ}{{\'L}}1    {ĺ}{{\'l}}1    {Ľ}{{\v{L}}}1  {ľ}{{\v{l}}}1  {Ł}{{\L{}}}1
      {ł}{{\l{}}}1   {Ń}{{\'N}}1    {ń}{{\'n}}1    {Ň}{{\v{N}}}1  {ň}{{\v{n}}}1
      {Ő}{{\H{O}}}1  {ő}{{\H{o}}}1  {Ŕ}{{\'{R}}}1  {ŕ}{{\'{r}}}1  {Ř}{{\v{R}}}1
      {ř}{{\v{r}}}1  {Ś}{{\'S}}1    {ś}{{\'s}}1    {Ş}{{\c{S}}}1  {ş}{{\c{s}}}1
      {Š}{{\v{S}}}1  {š}{{\v{s}}}1  {Ť}{{\v{T}}}1  {ť}{{\v{t}}}1  {Ũ}{{\~U}}1
      {ũ}{{\~u}}1    {Ū}{{\={U}}}1  {ū}{{\={u}}}1  {Ů}{{\r{U}}}1  {ů}{{\r{u}}}1
      {Ű}{{\H{U}}}1  {ű}{{\H{u}}}1  {Ų}{{\k{U}}}1  {ų}{{\k{u}}}1  {Ź}{{\'Z}}1
      {ź}{{\'z}}1    {Ż}{{\.Z}}1    {ż}{{\.z}}1    {Ž}{{\v{Z}}}1  {ž}{{\v{z}}}1
}
\usepackage{polski}
\usepackage[polish]{babel}


\newcounter{ImageCount}
\newcommand{\image}[1]{\textit{Rys. \arabic{ImageCount} -- #1} \stepcounter{ImageCount}}
\stepcounter{ImageCount}
\newcommand{\img}[3][1]{\begin{center}{\includegraphics[scale=#1]{#2}\\ \image{#3}}\end{center}}




\begin{document}

\begin{center}
\begin{tabular}{|c|c|}

\hline 
		Wydział Informatyki Politechniki Białostockiej & Data: 15.01.2024  \\
					   								   & Przetwarzanie języka naturalnego \\
\hline
		Projekt, Tworzenie podsumowań tekstu & Prowadzący: \\
		Piotr Zalewski, Kacper Świderek & dr inż. Tomasz Łukaszuk \\
\hline 

\end{tabular}
\end{center}

\section{Przedstawienie zadania projektowego}
Celem projektu było opracowanie prostego narzędzia do tworzenia 
podsumowań dokumentów tekstowych. Opracowane narzędzie daje możliwość 
wybrania metod/y użytej do tworzenia podsumowania oraz ew. porównania
jakości podsumowania na podstawie metryk BERTScore oraz ROUGE.
\section{Przedstawienie rozszerzenia zadania do oceny 5.0}
Rozszerzeniem zadania jest dodanie możliwości łączenia metod tworzących 
podsumowania, gdzie przy metodach ekstrakcyjnych metody łączone są
przy użyciu sumy ważonej wyników jakie każda metoda przypisuje do zdań
i następne wybranie N zdań z najwyższym wynikiem, z kolei w przypadku metod
abstrakcyjnych polega na zawężaniu obszerności podsumowania stopniowo
każdą kolejną metodą (kolejność podawana jest przez użytkownika). Przykładowo,
najpierw z tekstu A tworzone jest podsumowanie metodą X o wielkości 20\% tekstu 
pierwotnego, następnie wykorzystywana jest metoda Y, która na podstawie tekstu podsumowania
stworzonego za pomocą metody X, tworzy podsumowanie podsumowania obszerności 25\%
tego podsumowania. 
\section{Wprowadzenie teoretyczne (naukowe) do zagadnienia}
Przetwarzanie języka naturalnego to dziedzina skupiająca wiedzę z zakresu
językoznastwa oraz informatyki zajmująca się automatyzacją analizy, rozumienia,
tłumaczenia i generowania tekstu/języka naturalnego.
Podsumowanie tekstu polega na streszczeniu znaczenia tego tekstu, używając
innego tekstu, o mniejszym rozmiarze. Wtórnym do zagadnienia tworzenia podsumowań
jest zagadnienie analizy tekstu, czyli sposobu pozyskiwania i kodowania informacji o tekście
tj. występujące w nim słowa i semantyka. Wykorzystując metody kodowania informacji o tekście
można projektować algorytmy, które na podstawie tych reprezentacji generują podsumowania.
Wyróżnia się dwie grupy metod służących do tworzenia podsumowań, metody ekstrakcyjne,
polegające na wybraniu najbardziej znaczących zdań z tekstu oraz metody abstrakcyjne,
polegające na tworzeniu nowych zdań na podstawie tekstu. Z reguły, z racji na większą złożoność,
metody abstrakcyjne tworzą lepsze podsumowania, aczkolwiek ze względu na to, że tworzone
podsumowania to całkowicie nowy tekst, mogą zaistnieć w nim przekłamania tekstu pierwotnego. Przy metodach
ekstrakcyjnych wybierane są jedynie zdania kluczowe z tekstu, przez co są one w tym kontekście bezpieczniejsze.

\subsection{Metody ekstrakcyjne}
\subsubsection{Wybór na podstawie długości zdania}
Naiwną metodą typowania zdań do podsumowania jest tworzenie rankingu na podstawie
ilości słów w zdaniu (nie biorąc pod uwagę \textit{stop words} oraz różnych morfologii 
tych samych słów). Należy zauważyć, że przy tej metodzie ekstrakcyjnej podsumowanie
będzie zawsze możliwie nadłuższe.
\subsubsection{Wybór pierwszego i ostatniego zdania}
Kolejną naiwną metodą jest wybór pierwszego i ostatniego zdania bazując na założeniu, że
są one w większości przypadków podsumowywujące.
\subsubsection{TF-IDF}
TF-IDF to metoda reprezentacji tekstu oparta na bag-of-words. Pierwsza praca traktująca
o TF-IDF została opublikowana w 1972 \cite{sparckjones1972statistical}. Przy tej metodzie tekst 
reprezentowany jest jako macierz której każdy wiesz odpowiada danemu zdaniu z tekstu, a 
każda kolumna odpowiada danemu słowu. Wartości macierzy wyliczane są za pomocą wzoru (1).

\begin{center}
	\begin{equation}
		w_{ij} = tf_{ij} \cdot log(\frac{N}{df_i})
	\end{equation}
\end{center}

Gdzie w przypadku opisywanego rozwiązania $tf_{ij}$ oznacza ilość wystąpień słowa
$i$ w zdaniu $j$, N oznacza ilość zdań, a $df_i$ oznacza ilość zdań zawierających
słowo $i$. 
Przy tworzeniu podsumowania tworzony jest ranking w którym dla każdego zdania sumowane
są otrzymane wartości $w_{ij}$ zgodnie ze wzorem (2). Następnie wybierane jest $K$ 
zdań z najwyższym wynikiem.

\begin{center}
	\begin{equation}
		S_{j} = \sum_{i = 0}^{M} w_{ij}
	\end{equation}
\end{center}

\subsubsection{TextRank}
TextRank to algorytm bazujący na algorytmie PageRank opublikowany w 2004 roku \cite{mihalcea2004textrank}.
W algorytmie tworzony jest graf którego wierzchołki to, jak w opisywanym rozwiązaniu, zdania, słowa
kluczowe lub inne jednostki tekstu. W przypadku zdań krawędzie pomiędzy wierzchołkami reprezentują 
podobieństwo pomiędzy zdaniami. Podobieństwo może być określane chociażby za pomocą ilości wspólnych
słów/tokenów bądź przy użyciu podobieństwa wektorów wziętych np. z macierzy stworzonej za pomocą
algorytmu TF-IDF. Następnie przechodząc po grafie wylicza się wynik każdego wierzchołka według wzoru (3) \cite{mihalcea2004textrank}.

\begin{center}
	\begin{equation}
		WS(V_i) = 0.15 + 0.85 \cdot \sum_{V_j\in In(V_i)}^{} \frac{w_{ij}}{\sum_{V_k\in Out(V_j)}^{} w_{jk}} \cdot WS(V_j)
	\end{equation}
\end{center}

Jak w algorytmie PageRank we wzorze (3) dany wierzchołek otrzymuje tym wyższy wynik im więcej
wierzchołków nań wskazuje. Co więcej, wynik więrzchołków wskazujących na ten wierzchołek również
ma wpływ na wynik tego wierzchołka. We wzorze (3) dodatkowo dochodzą wspomniane wagi krawędzi, 
im wyższa waga tym większe podobieństwo pomiędzy tymi wierzchołkami tym większa kontrybucja 
do końcowego wyniku. Sortując wartości wierzchołków otrzymuje się listę zdań uporządkowaną malejąco
według stopnia reprezentatywności przez nie całego tekstu.

\subsection{Metody abstrakcyjne}

\subsubsection{T5}
T5 to model transformatorowy opracowany przez Google, który został zaprojektowany do obsługi szerokiego zakresu zadań NLP (przetwarzania języka naturalnego) w jednolity sposób.
Kluczową ideą T5 jest potraktowanie wszystkich zadań NLP jako zadań tekst-do-tekstu. Oznacza to, że zarówno zadania klasyfikacji, jak i generowania tekstu są formułowane jako zadania, w których model przyjmuje tekst wejściowy i generuje tekst wyjściowy.
Aby to osiągnąć, T5 wykorzystuje specjalne "prefiksowe" tokeny (np. "tłumaczenie na język niemiecki:", "streszczenie:") do odróżniania różnych zadań.
T5 bazuje na architekturze transformatora z enkoderem i dekoderem.
Wykorzystuje mechanizm uwagi (attention) i trenowany jest na ogromnym zbiorze danych tekstowych.
Model posiada różne warianty rozmiarowe, co pozwala na dostosowanie go do dostępnych zasobów obliczeniowych i konkretnych zadań \cite{DBLP:journals/corr/abs-1910-10683}.

\subsubsection{BART}
BART (Bidirectional and Auto-Regressive Transformers) to model zaproponowany przez Facebook AI w 2019 roku.
BART to model sekwencyjny do sekwencji (seq2seq), który został zaprojektowany do zadań wymagających zarówno kodowania kontekstu (rozumienia), jak i generowania tekstu.
Jest on często stosowany w zadaniach takich jak streszczanie tekstu, generowanie dialogów i tłumaczenie maszynowe. 
BART łączy cechy dwukierunkowego kodera (podobnego do BERT) z autoregresyjnym dekoderem (podobnym do modeli GPT).
BART składa się z kodera transformatorowego i dekodera transformatorowego. 
Koder jest dwukierunkowy i służy do kodowania kontekstu wejściowego. 
Dekoder jest autoregresyjny i generuje tekst wyjściowy token po tokenie, na podstawie zakodowanego kontekstu \cite{DBLP:journals/corr/abs-1910-13461}.

\subsubsection{Pegasus}
Pegasus (Pre-training with Extracted Gap-sentences for Abstractive Summarization) to model zaproponowany przez Google Research w 2020 roku. 
Pegasus to model pre-treningowy, który jest specjalnie zaprojektowany do streszczania abstrakcyjnego. 
Jego unikalna metoda pre-treningu opiera się na "maskowaniu" całych zdań z dokumentu i uczeniu się ich odtwarzania na podstawie pozostałego tekstu. 
Celem tego podejścia jest nauczenie modelu generowania streszczeń poprzez wyodrębnianie najważniejszych informacji i ich ponowne przedstawianie w skróconej formie.
Pegasus bazuje na architekturze transformatora sekwencja-do-sekwencji. Posiada enkoder i dekoder, podobnie jak BART, ale jego pre-trening jest dostosowany do streszczania \cite{DBLP:journals/corr/abs-1912-08777}.

\subsection{Ocena jakości podsumowań}
Najbardziej popularnymi metodami oceny jakości podsumowań jest \textit{ROUGE score} oraz
\textit{BARTScore}. Jakość metryki oceniana jest na podstawie jej korelacji z oceną
subiektywną (oceną dokonaną przez człowieka). Najlepszą metodą oceny podsumowania jest 
ocena subiektywna ale ze względu na to, że ręczne tworzenie podsumowań jest pracochłonne oraz że
ocena jest niestabilna (ciężko porównać dwa podsumowania i określić które z nich jest lepsze
i o ile), dąży się do wypracowania technik jak te wspomniane wyżej. 

\subsubsection{ROUGE score}
ROUGE score to metoda zaproponowana po raz pierwszy w 2004 roku \cite{lin2004rouge}. Jak już wspomniano,
przy ocenie jakości, podsumowanie dokonane ręcznie przez człowieka jest brane jako podsumowanie
wzorcowe. \textit{ROUGE score} opiera się na zliczaniu nakładających się jednostek tekstu takich jak
słowa, zdania czy n-gramy podsumowania wygenerowanego z podsumowaniem wzorcowym. Praca \cite{lin2004rouge}
wprowadza trzy miary \textit{ROUGE score}
\begin{itemize}
	\item \textit{ROUGE-N} - stosunek liczby n-gramów występujących zarówno w wygenerowanym podsumowaniu jak i w 
		podsumowaniach wzorcowych do sumy n-gramów z podsumowań wzorcowych.
	\item \textit{ROUGE-L} - na podstawie sumy najdłuższych wspólnych sekwencji wyliczanie 3 miar, 
		\textit{recall}, \textit{precision} i \textit{F}. Gdzie każda najdłuższa wspólna sekwencja jest
		wyliczana z danego zdania podsumowania referencyjnego i wszystkimi zdaniami z podsumowania wygenerowanego.
		Miary \textit{recall} i \textit{precision} to wspomniana suma podzielona przez kolejno długość podsumowania
		referencyjnego i wygenerowanego. Miarę \textit{F} określa się wzorem (4), gdzie $\beta$ to zmienna 
		kontrolująca czy \textit{recall} ($\beta > 1$) czy \textit{precision} ($\beta < 1$) jest bardziej 
		istotne w obliczonej wartości.
		\begin{center}
			\begin{equation}
				F = \frac{(1 + \beta^2) \cdot recall\cdot precision}{recall + \beta^2 \cdot precision}
			\end{equation}
		\end{center}
	\item \textit{ROUGE-W} - Ważona wersja \textit{ROUGE-L} biorąca pod uwagę
		ciągłość wspólnych sekwencji. Tym większa waga wspólnej sekwencji im większa jest jej ciągłość (\textit{ROUGE-L} ignoruje tą właściwość przez co dwie
		wykryte wspólne sekwencje, jedna ciągła a druga nie, mogą zostać ocenione
		tak samo mimo, że oczywistym jest, że ciągła sekwencja jest bardziej
		porządana).
	\item \textit{ROUGE-S} - Wyliczana jest liczba nakładających się \textit{skip-bigrams} 
		czyli par słów o dowolnej odległości w zdaniu ale występujących po sobie. Na podstawie tej 
		liczby podobnie jak w przypadku \textit{ROUGE-N} i \textit{ROUGE-L} wyliczane są
		\textit{recall}, \textit{precision} i \textit{F}.
\end{itemize}
Warto zaznaczyć, że miara ROUGE może negatywnie ocenić podsumowanie wygenerowane,
nawet gdy jest ono dobrej jakości, np. gdy zawiera ono podobną treść
co w podsumowaniu wzorcowym ale napisaną w inny sposób, używając innych słów itd.


\subsubsection{BARTscore}
BERTScore to metryka zaproponowana w 2019 roku \cite{zhang2020bertscoreevaluatingtextgeneration}. 
W metodzie tej używa się modelu BART do oceny podsumowania wygenerowanego na podstawie podsumowania
referencyjnego.

\section{Opis realizacji technicznej (wykorzystane biblioteki, dane treningowe)}
\subsection{Biblioteki}
Do wykonania zadania skorzystano z następujących bibliotek:
\begin{itemize}
	\item spacy - segmentacja tekstu, TextRank
	\item sklearn - TF-IDF
	\item transformers - abstrakcyjne podsumowania, BART, Pegasus, T5
	\item rouge\_score - wyliczanie rouge\_score podsumowania.
	\item bert\_score - wyliczanie bert\_score podsumowania.
\end{itemize}
\subsection{Dane treningowe}
Dane treningowe wytworzono korzystając z biblioteki gutenberg \cite{gutenberg}
zawierającej w pełni darmowe cyfrowe kopie książek tj. \textit{War and peace}, którym
posłużono się do tworzenia podsumowań. Podsumowania wzorcowe tworzono w większości przy pomocy
\textit{ChatGPT} ale również ręcznie.

\section{Instrukcja korzystania z programu}
\section{Przykłady użycia programu (dane wejściowe i otrzymane wyniki)}

\bibliographystyle{plain}
\bibliography{refs}

\end{document}
